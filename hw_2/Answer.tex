\documentclass[14pt]{article}
\usepackage{amsmath,amssymb}
\usepackage{dsfont}
\usepackage{setspace}
\usepackage{gensymb}
\usepackage{graphicx}
\usepackage[margin=2.5cm]{geometry}
\usepackage{enumerate}
\usepackage{float}
\usepackage{multirow}
\setlength{\parindent}{2cm}
\setlength{\parskip}{0.25em}
\usepackage[utf8]{inputenc}
\usepackage{mathtools}
\usepackage{mathrsfs}
\usepackage{titlesec}
\usepackage{indentfirst}
\usepackage{pdfpages}

\renewcommand{\d}{\text{d}}
\newcommand{\D}{\text{D}}

\newcommand{\dbar}{\d\hspace*{-0.2em}\bar{}\hspace*{0.3em}}
\newcommand{\deltabar}{\boldsymbol{\delta\hspace*{-0.2em}\bar{}\hspace*{0.3em}}}

\title{\Large{PHYS 239   HW 2}}
\author{Hongrui Li}
\date{October 2020}

\begin{document}

\maketitle
\begin{enumerate}
    \item Since the cloud has depth $D=100 \text{pc}=3.0856775814914\times 10^{18} \text{cm}$, then the column density reads $N=nD=3.0856775814914\times 10^{18} \text{cm}^{-2}$. Since the optical depth can be written in
    $$\tau_\nu=N\sigma_\nu$$
    With $\tau_\nu$ specified, we can rewrite the equation: $\sigma_\nu=\frac{\tau_\nu}{N}$ so
    \begin{enumerate}
        \item $\tau_\nu=10^{-3}\implies \sigma_\nu=\frac{\tau_\nu}{N}=3.24\times 10^{-22}\text{cm}^2$
        
        \item $\tau_\nu=1\implies \sigma_\nu=\frac{\tau_\nu}{N}=3.24\times 10^{-19}\text{cm}^2$
        
        \item $\tau_\nu=10^{3}\implies \sigma_\nu=\frac{\tau_\nu}{N}=3.24\times 10^{-16}\text{cm}^2$
    \end{enumerate}

    \item Everything else: in program 
    
    For convenience, I set $S=1$ and write everything in terms of optical depth (or set $N=1$)
\end{enumerate}

\end{document}
